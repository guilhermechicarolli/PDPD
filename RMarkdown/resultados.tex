% Options for packages loaded elsewhere
\PassOptionsToPackage{unicode}{hyperref}
\PassOptionsToPackage{hyphens}{url}
%
\documentclass[
  11pt,
]{article}
\usepackage{lmodern}
\usepackage{setspace}
\usepackage{amssymb,amsmath}
\usepackage{ifxetex,ifluatex}
\ifnum 0\ifxetex 1\fi\ifluatex 1\fi=0 % if pdftex
  \usepackage[T1]{fontenc}
  \usepackage[utf8]{inputenc}
  \usepackage{textcomp} % provide euro and other symbols
\else % if luatex or xetex
  \usepackage{unicode-math}
  \defaultfontfeatures{Scale=MatchLowercase}
  \defaultfontfeatures[\rmfamily]{Ligatures=TeX,Scale=1}
\fi
% Use upquote if available, for straight quotes in verbatim environments
\IfFileExists{upquote.sty}{\usepackage{upquote}}{}
\IfFileExists{microtype.sty}{% use microtype if available
  \usepackage[]{microtype}
  \UseMicrotypeSet[protrusion]{basicmath} % disable protrusion for tt fonts
}{}
\usepackage{xcolor}
\IfFileExists{xurl.sty}{\usepackage{xurl}}{} % add URL line breaks if available
\IfFileExists{bookmark.sty}{\usepackage{bookmark}}{\usepackage{hyperref}}
\hypersetup{
  hidelinks,
  pdfcreator={LaTeX via pandoc}}
\urlstyle{same} % disable monospaced font for URLs
\usepackage[margin=1in]{geometry}
\usepackage{graphicx,grffile}
\makeatletter
\def\maxwidth{\ifdim\Gin@nat@width>\linewidth\linewidth\else\Gin@nat@width\fi}
\def\maxheight{\ifdim\Gin@nat@height>\textheight\textheight\else\Gin@nat@height\fi}
\makeatother
% Scale images if necessary, so that they will not overflow the page
% margins by default, and it is still possible to overwrite the defaults
% using explicit options in \includegraphics[width, height, ...]{}
\setkeys{Gin}{width=\maxwidth,height=\maxheight,keepaspectratio}
% Set default figure placement to htbp
\makeatletter
\def\fps@figure{htbp}
\makeatother
\setlength{\emergencystretch}{3em} % prevent overfull lines
\providecommand{\tightlist}{%
  \setlength{\itemsep}{0pt}\setlength{\parskip}{0pt}}
\setcounter{secnumdepth}{-\maxdimen} % remove section numbering

\author{}
\date{\vspace{-2.5em}}

\begin{document}

\setstretch{1.25}
\clearpage

\hypertarget{resultados}{%
\section{Resultados}\label{resultados}}

~~~~O algoritmo Maxent demonstrou boa performance preditiva, obtendo
altos valores médios de AUC (\protect\hyperlink{apuxeandice}{gráficos 8
e 9}). A proporção de área potencial ganha e perdida sob os diferentes
cenários climáticos com relação ao presente variou entre ambas as
espécies (\label{tabela6}), todas apresentaram contração na área
ambiental adequada nos cenários futuros de mudanças climáticas.

Os modelos previram redução de 36.98\% de áreas adequadas (com respeito
à área presente) para \emph{Lonchophylla bokermanni} no cenário de RCP
4.5 para o ano de 2050 e diminuição de 58.06\% de área para o panorama
de RCP 8.5, para 2050 (\protect\hyperlink{apuxeandice}{tabela 7}). A
mesma tendência foi observada para \emph{Encholirium subsecundum}, a
qual previmos encolhimento de 72.70\% de área apropriada (relativa à
área do presente) para o futuro de RCP 4.5 (2050), enquanto que no
cenário de RCP 8.5 a redução na área é de 81.11\% (com respeito ao
presente).

Previmos uma diminuição na área de sobreposição potencial entre a
distribuição de ambas as espécies (planta+morcego), a qual diminui
67.76\% no cenário de RCP 4.5 e 79.86\% no RCP 8.5, em relação à
sobreposição no presente (\protect\hyperlink{apuxeandice}{tabela 8}). A
taxa de \emph{mismatch} espacial (desencontro geográfico) entre a planta
e o morcego aumentou com relação à distribuição de \emph{Lonchophylla
bokermanni}, que apresentou 26.07\% de sua distribuição potencial
presente sem sobreposição com a da planta, no cenário de RCP 4.5,
61.56\% da área do morcego não apresentou sobreposição e no RCP 8.5,
63.07\% de sua área. Em contrapartida, observamos diminuição na taxa de
\emph{mismatch} para a planta, previmos 28.06\% de sua área presente não
tenha sobreposiçãp com a distribuição do morcego, 15.07\% da área no
futuro RCP 4.5 sem sobreposição e 23.30\% da área no RCP 8.5
(\protect\hyperlink{apuxeandice}{tabela 9}).

\end{document}
